\documentclass[10pt]{article}
\usepackage[a4paper, total={6in, 11in}]{geometry}

\author{Miguel P Xochicale, Chris Baber \\
% \author{Miguel P Xochicale \qquad Chris Baber \\
\{map479, c.baber\}@bham.ac.uk \\
School of Engineering\\
Department of Electronic, Electric and System Engineering\\
University of Birmingham, UK}
%%%%%%%%%%%%%%%%%%%%%%%%%%%
% \title{The Inherent Chaos in Human Movement} ()
%%%%%%%%%%%%%%%%%%%%%%%%%%%
% \title{The Inherent Chaos in the Variability Human Movement and Facial
% Expressions} %(Monday, 20th November 2017)
%%%%%%%%%%%%%%%%%%%%%%%%%%%
% \title{Uniform and Nonuniform Time-Delay Embedding Methodologies for Movement
% and Emotion Variability} %(Monday, 21th November 2017)
%%%%%%%%%%%%%%%%%%%%%%%%%%%
% \title{The Inherent Chaos in Human Movement and Emotions}
%(Tuesday, 22th November 2017)
%%%%%%%%%%%%%%%%%%%%%%%%%%%
% \title{Quantifying the Inherent Chaos in Human Movement and Emotions}
%%%%%%%%%%%%%%%%%%%%%%%%%%%
%(Friday, 8th December 2017)
\title{Quantifying the Inherent Chaos of \\ Human Movement Variability}
%(Monday, 18th December 2017)
%%%%%%%%%%%%%%%%%%%%%%%%%%%%
%%\title{Quantifying the Inherent Chaos of Human Movement Variability 
%in the context of Human-Humanoid Imitation Activities.}
%%(Wednesday, 10th January 2018)
%%%%%%%%%%%%%%%%%%%%%%%%%%%%

\date{\today}

\begin{document}
\maketitle

%%%%%%%%%%%%%%%%%%%%%%%%%%%%%%%%%%%%%%%%%%%%%%%%%%%%%%%%%%%%%%%%%%%%%%%%%%%%%%%%
% Abstracts should contain less than 500 words 
% NUMBER OF WORDS: 

%%%%%%%%%%%%%%%%%%%%%%%%%%%%%%%%%%%%%%%%%%%%%%%%%%%%%%%%%%%%%%%%%%%%%%%%%%%%%%%%
%%%%% One or two sentences proving a basic introduction to the field,
%%%%% comprehensible to a scientist in any discipline.
Movement variability is defined as the variations that occur in motor
performance across multiple repetitions of a task and such behaviour is an 
inherent feature within and between each persons' movement \cite{tlockhart2013}.
In the previous three decades, research on measurement and understanding of
movement variability with methodologies of nonlinear dynamics has been well established 
in areas such as biomechanics, sport science, psychology, cognitive science, 
neuroscience and robotics \cite{stergiou2011,harbourne2009}.
%%%%%%%%%%%%%%%%%%%%%%%%%%%%%%%%%%%%%%%%%%%%%%%%%%%%%%%%%%%%%%%%%%%%%%%%%%%%%%%%
%%%%% Two to three sentences of more detailed background,
%%%%% comprehensible to scientist in related disciplines.
To quantify movement variability, we therefore consider a methodology from nonlinear dynamics
called uniform reconstructed state space where essentially dynamics of
an unknown system can be reconstructed using one dimensional time series.
As pointed out by Bradley et al. \cite{bradley2015} uniform reconstructed state space,
if done right, can guarantee to be topologically identical to the true dynamics
and determine dynamics invariants such as fractal dimension, Kolmogorov-Sinai
entropy or Lyaponov exponents.
These algorithms, however, require time series measured with costly sensors 
that provide well sampled data with little noise.
Such requirement is generally a common problem when doing precise
characterisation of time series using dynamic invariants,
to which Bradley et al. \cite{bradley2015} proposed additional tools,
for practitioners, of nonlinear time series analysis such as surrogate data, 
permutation entropy, recurrence plots and network characteristics for time series.
%%%%%%%%%%%%%%%%%%%%%%%%%%%%%%%%%%%%%%%%%%%%%%%%%%%%%%%%%%%%%%%%%%%%%%%%%%%%%%%%
%%%%% One sentence clearly stating the general problem being addressed by this
%%%%% particular study.
For this study, we are interested in the use of uniform reconstructed
state space and the analysis of recurrence plots so as to understand 
the quantification of movement variability.
Particularly, we are interested in the analysis of data collected through
cheap wearable inertial sensors and its effects on the 
reconstructed state space and the recurrence plots
for different lengths and preprocessing techniques 
(like smoothing and normalisation) of the time series.
%%%%%%%%%%%%%%%%%%%%%%%%%%%%%%%%%%%%%%%%%%%%%%%%%%%%%%%%%%%%%%%%%%%%%%%%%%%%%%%%
%%%%% One sentence summarising the main result (with the words "here we show"
%%%%% or their equivalent)
So, here we show the characterisation of human movement variability in the
context of human-humanoid imitation activities.
%%%%%%%%%%%%%%%%%%%%%%%%%%%%%%%%%%%%%%%%%%%%%%%%%%%%%%%%%%%%%%%%%%%%%%%%%%%%%%%%
%%%%% Two or three sentences explaining what the main results reveals in direct
%%%%% comparison to what was thought to be the case previously, or how the main
%%%%% results adds to previous knowledge.
Specifically, we explore the reconstruction of state spaces and its recurrence plots for
20 participants performing repetitions of simple vertical and horizontal arm 
movements in normal and faster velocity. 
We also explore the differences between wearable inertial sensors attached 
to the person and to the humanoid robot and between different axes of inertial sensors. 
With that in mind, our contribution to knowledge is in regard to the 
reliability of data from cheap wearable inertial sensors 
to analyse human movement in the context of human-humanoid imitation activities 
using methodologies of nonlinear dynamics.
%%%%%%%%%%%%%%%%%%%%%%%%%%%%%%%%%%%%%%%%%%%%%%%%%%%%%%%%%%%%%%%%%%%%%%%%%%%%%%%%
%%%%% One or two sentences to put the results into a more general context.
Such understanding and measurement of movement variability using 
cheap wearable inertial sensors lead us to have a more intuitive selection of parameters
to reconstruct the state spaces and to create meaningful interpretations 
of the recurrence plots. Additionally, having a better understanding of 
nonlinear dynamics tools with the use of cheap inertial sensors is important 
for the development of better diagnostic tools for various pathologies which 
can be applied in areas of rehabilitation, entertainment or sport science \cite{tlockhart2013}.
%%%%%%%%%%%%%%%%%%%%%%%%%%%%%%%%%%%%%%%%%%%%%%%%%%%%%%%%%%%%%%%%%%%%%%%%%%%%%%%%
%%%%% Two or three sentences to provide a broader perspective, readily comprehensible
%%%%% to a scientist in any discipline, may be included in the first paragraph
%%%%% if the editor considers that the accessibility of the paper is significantly
%%%%% enhanced by their inclusion. Under this circumstances, the length of the
%%%%% paragraph can be up to 300 words



\begin{thebibliography}{10}
\bibitem{stergiou2011}
Nicholas Stergiou and Leslie M. Decker,
{\it Nonlinear dynamics, and pathology: Is there a connection?},
{Human Movement Science vol.30, no.5} (2011)

\bibitem{harbourne2009}
Regina T Harbourne and Nicholas Stergiou,
{\it Movement Variability and the Use of Nonlinear Tools: Principles to Guide
Physical Therapist Practice}, {Physical Therapy, vol.89 no.3}  (2009)

\bibitem{bradley2015}
Elizabeth Bradley and Holger Kantz,
{\it Nonlinear time-series analysis revisited}, {Chaos: An Interdisciplinary
Journal of Nonlinear Science, vol.25 no.9}  (2015)


\bibitem{tlockhart2013}
Thurmon Lockhart and Nick Stergiou,
{\it New Perspectives in Human Movement Variability},
{Annals of Biomedical Engineering, vol 41 no.8} (August 2013)


\end{thebibliography}


\end{document}
