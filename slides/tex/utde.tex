
\section{UTDE}

\subsection{}
%%%%%%%%%%%%%%%%%%%%%%%%%%%%%%%%%%%%%%%%%%%%%%%%%%%%%%%%
{
\paper{Takens F. 1981 in {\bf Dynamical Systems and Turbulence}}

\begin{frame}{Reconstructed State Space Theorem}
    \begin{figure}
        \includegraphicscopyright[width=0.65\linewidth]{utde/utde}{Copyright (Xochicale 2018) }
	%\caption{Reconstructed State Space} 
   \end{figure}
	
\end{frame}
}


\subsection{}
%%%%%%%%%%%%%%%%%%%%%%%%%%%%%%%%%%%%%%%%%%%%%%%%%%%%%%%%
{
\paper{Takens F. 1981 in {\bf Dynamical Systems and Turbulence}}

\begin{frame}{THE METHOD OF Uniform Time-Delay Embedding (UTDE)}

For a given discrete time series $x(n) = [x(1) , x(2), \dots, x(N)]$,
a reconstructed state space matris is defined as 
\begin{eqnarray*}
 \mathbf{X^{m}_{\tau}}
=
    = \begin{pmatrix} \nonumber
      x(n)  \\
      x(n-\tau) \\
      \vdots  \\
      x(n-(m-1)\tau) \\
      \end{pmatrix}^T
% [ x(n), \\ 
%	x(n - \tau), \\
%	x(n-2\tau), \dots , x (n-(m-1)\tau) ]
\end{eqnarray*}

%which creates a concatenated column-wise matrix of time-delay versions of the original signal:
%\begin{equation}
%  \resizebox{\textwidth}{!}{$\displaystyle
%  \mathbf{X}
%    = \begin{pmatrix} \nonumber
%      x(1) & x(1 - \tau) & x(1-2\tau) & \dots & x (1-(m-1)\tau) \\
%      x(2) & x(2 - \tau) & x(2-2\tau) & \dots & x (2-(m-1)\tau) \\
%      \vdots &  &  & \ddots & \vdots \\
%      x(N) & x(N - \tau) & x(N-2\tau) & \dots & x (N-(m-1)\tau) \\
%      \end{pmatrix}
%     $}
%\end{equation}

where $m$ is the \textbf{ embedding dimension}  and  $\tau$ is the \textbf{ embedding delay}.




\end{frame}
}


\subsection{}
%%%%%%%%%%%%%%%%%%%%%%%%%%%%%%%%%%%%%%%%%%%%%%%%%%%%%%%%
{
\paper{Kabiraj et al. 2012 in {\bf Chaos: An Interdisiplinary Journal of Nonlinear Science}}

\begin{frame}{Average Mutual Information (AMI)}
    \begin{figure}
        \centering
        \includegraphicscopyright[width=0.7\linewidth]{utde/ami}{Copyright Xochicale 2014}
	\caption{(A, B) AMI values for (C) chaotic and (D) noise time series.} 
   \end{figure}
	
\end{frame}
}


\subsection{}
%%%%%%%%%%%%%%%%%%%%%%%%%%%%%%%%%%%%%%%%%%%%%%%%%%%%%%%%
{
\paper{Cao L. 1997 in {\bf Physica D}}

\begin{frame}{False Nearest Neighbours (FNN)}
    \begin{figure}
        \centering
        \includegraphicscopyright[width=1.0\linewidth]{utde/fnn}{Copyright Xochicale 2014}
	\caption{(A,B) $E_1(m)$ and (C, D) $E_2(m)$ values for (E) chaotic 
		and (F) random time series} 
   \end{figure}
	
\end{frame}
}


