

\section{Conclusions and Future Work}

\subsection{}
%%%%%%%%%%%%%%%%%%%%%%%%%%%%%%%%%%%%%%%%%%%%%%%%%%%%%%%%
{
\begin{frame}{Conclusions}

\begin{itemize}
	\item (+) RSS and RQA can potentially be used to quantify arm movement variability.
However,
	\item (-) RSS and RQA are sensitive to window length, embedding parameters or thresholds.
\end{itemize}

%\badge{/badge/badge_v00}
\end{frame}
}

\subsection{}
%%%%%%%%%%%%%%%%%%%%%%%%%%%%%%%%%%%%%%%%%%%%%%%%%%%%%%%%
{
%\paper{Xochicale et al. 2018 in Progress}


\begin{frame}{Future Work}

\begin{itemize}
	\item Test other techniques of Nonlinear Dynamics, e.g. Lyapunov Exponents, Poincare Maps.
	\item Use of Convolutional Neural Networks for automatic identification of Movement Variability
\end{itemize}

\vspace{-1cm}

    \begin{figure}
        %\centering
        \includegraphicscopyright[width=0.9\linewidth]{conclusion_futurework/futurework}{Work in progress (Xochicale M. et al. 2018) }
	\caption{Pilot experiment for quantification of MV for complex movements and changes in facial expressions.} 
   \end{figure}
	
\end{frame}
}



