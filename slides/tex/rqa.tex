
\section{Recurrence Quantification}

\subsection{}
%%%%%%%%%%%%%%%%%%%%%%%%%%%%%%%%%%%%%%%%%%%%%%%%%%%%%%%%
{
\paper{Eckmann et al. 1987 in {\bf Europhysics Letters}}

\begin{frame}{Recurrence Plots}


$\mathbf{R}^{m}_{i,j} (\epsilon)$ is two dimensional plot of $N \times N$ square matrix
defined by

%%********************************[EQUATION]************************************
\begin{equation}
\mathbf{R}^{m}_{i,j} (\epsilon) = 
\Theta ( \epsilon_i - || X(i) - X(j) || ), \quad X(i) \in \mathbb{R}^m, \quad i,j=1,\dots,N
\end{equation}

%%********************************[EQUATION]************************************
where $N$ is the number of cosidered states of $X(i)$, 
$\epsilon$ is a theshold distance, 
$ || \dot ||$ a norm, and $\Theta( \dot )$ is the Heaviside function.

\end{frame}
}



\subsection{}
%%%%%%%%%%%%%%%%%%%%%%%%%%%%%%%%%%%%%%%%%%%%%%%%%%%%%%%%
{
\paper{Eckmann et al. 1987 in {\bf Europhysics Letters}}

\begin{frame}{Recurrence Plots}
    \begin{figure}
        \includegraphicscopyright[width=0.8\linewidth]{rqa/rp}
		{Figure is adapted from (Marwan et al. 2007). 
		%R code to reproduce figure is available from (Xochicale 2018).
		 }
	\caption{(A) State space for Lorenz systems, and 
		(B) Recurrence plots with no embeddings and $\epsilon=5$} 
   \end{figure}


\end{frame}
}

\subsection{}
%%%%%%%%%%%%%%%%%%%%%%%%%%%%%%%%%%%%%%%%%%%%%%%%%%%%%%%%
{
\paper{Marwan et al. 2007 in {\bf Physics Reports}}

\begin{frame}{Recurrence Plot Patterns}
    \begin{figure}
        \includegraphicscopyright[width=\linewidth]{rqa/rpp}{Figure is adapted from (Marwan et al. 2007)}
	\caption{Recurrence plots for (A) uniformly distributed noise,
		(B) super-positionet harmonic oscillation,
		(C) drift logistic map with a linear increase term, and
		(D) disrupted brownian motion.
		} 
   \end{figure}
	
\end{frame}
}



\subsection{}
%%%%%%%%%%%%%%%%%%%%%%%%%%%%%%%%%%%%%%%%%%%%%%%%%%%%%%%%
{
\paper{Marwan et al. 2007 in {\bf Physics Reports}}

\begin{frame}{Recurrence Quantification Analysis}

\begin{description}
\item [ \textbf{REC} ] the percent of recurrence enumerates 
			the black dots in the RP excluding the line of identity.
\item [ \textbf{DET} ] the percent of determinism if the fraction of recurrence points
			that form diagonal lines. \\
			\textit{(interpreted as the predictibility where, for example,
				periodic signals show longer diagonal lines 
				than chaotic ones.
				)}
\item [ \textbf{RATIO} ] is the radio of DET to REC. \\
			\textit{(useful to discover dynamic transitions)}.
\item [ \textbf{LAM} ] computes the recurrence points in the vertical lines \\
			\textit{(analogous to DET)}.
\end{description}


	
\end{frame}
}






