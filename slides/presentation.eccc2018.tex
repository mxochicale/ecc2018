\documentclass[compress]{beamer}

%--------------------------------------------------------------------------
% Common packages
%--------------------------------------------------------------------------
\usepackage[english]{babel}
\usepackage{pgfpages} % required for notes on second screen
\usepackage{graphicx}
\usepackage{subfigure}
\usepackage{multicol}
\usepackage[normalem]{ulem}

\usepackage{tabularx,ragged2e}
\usepackage{booktabs}
\usepackage{marvosym}

\usepackage{fontawesome}



\usepackage{fontawesome}
% \usepackage[tt=false, type1=true]{libertine}
% \usepackage[T1]{fontenc}



\usepackage{tikz}
\usetikzlibrary{calc,shapes,shadows}
\usetikzlibrary{fadings}



%--------------------------------------------------------------------------
% Load theme
%--------------------------------------------------------------------------
\usetheme{mx}
\graphicspath{{figs/}}




%--------------------------------------------------------------------------
% General presentation settings
%--------------------------------------------------------------------------
\title{
	Quantifying the Inherent Chaos of \\ 
	Human Movement Variability
	} 
\subtitle{EEEC2018}
\date{Madrid, Spain, 4-7 June 2018}

\author{Miguel P Xochicale and Chris Baber}
\institute{School of Engineering \\{\bf University of Birmingham}}




%--------------------------------------------------------------------------
% Notes settings
%--------------------------------------------------------------------------
%\setbeameroption{show notes on second screen}
%\setbeameroption{hide notes}






\begin{document}


%%%%%%%%%%%%%%%%%%%%%%%%%%%%%%%%%%%%%%%%%%%%%%%%%%%%%%%%
\licenseframe{https://github.com/mxochicale/eccc2018}

%%%%%%%%%%%%%%%%%%%%%%%%%%%%%%%%%%%%%%%%%%%%%%%%%%%%%%%%
\maketitle



%\begin{frame}{Overview}
%\tableofcontents
%\end{frame}
%

%--------------------------------------------------------------------------
% Content
%--------------------------------------------------------------------------
%
\section{Movement Variability}


\subsection{}
%%%%%%%%%%%%%%%%%%%%%%%%%%%%%%%%%%%%%%%%%%%%%%%%%%%%%%%%
{

%\paper{Newell K M, Corcos D M, {\bf Variability and Motor Control}, 1993}
\paper{Lockhart T, Stergiou N, {\bf New Perspectives in Human Movement Variability}, 2013}

\begin{frame}{What is Movement Variabily?}

\LARGE
MOVEMENT VARIABILITY is defined as the variations that occur in motor
performance across multiple repetitions of a task and such behaviour is 
an inherent feature within and between each person's movement.
\end{frame}



}



%\section{State Space Reconstruction}
%
%%\subsection{}
%%%%%%%%%%%%%%%%%%%%%%%%%%%%%%%%%%%%%%%%%%%%%%%%%%%%%%%%%
%\imageframe[caption=Uniform Time-Delay Embedding]{utde/utde}


%%%%%%%%%%%%%%%%%%%%%%%%%%%%%%%%%%%%%%%%%%%%%%%%%%%%%%%%
\subsection{Images 1/2}
\begin{frame}{Picture with credit line}
    \begin{figure}
        \centering
        \includegraphicscopyright[width=\linewidth]{utde/utde}{Copyright EPFL 2014}
	\caption{Reconstructed State Space} 
   \end{figure}
	
\end{frame}



\section{Conclusions and Future Work}

\subsection{}
%%%%%%%%%%%%%%%%%%%%%%%%%%%%%%%%%%%%%%%%%%%%%%%%%%%%%%%%
{
\begin{frame}{Conclusions Future Work}

\begin{itemize}
	\item (+) Quantification for Arm Movement and Head Pose Estimation Variability with Nonlinear Dynamics is possible. However,
	\item (-) the timeseries from the landmarks are mounted on the pose location of the head. 
\end{itemize}

\begin{itemize}
	\item Test other techniques of Nonlinear Dynamics, e.g. Lyapunov Exponents, Recurrent Quantification Analysis
	\item Use of Convolutional Neural Networks for automatic identification of Movement Variability
\end{itemize}



\badge{/badge/badge_v00}
\end{frame}
}






%%%%%%%%%%%%%%%%%%%%%%%%%%%%%%%%%%%%%%%%%%%%%%%%%%%%%%%%
\begin{frame}{Bibliography}
    \begin{thebibliography}{10}

\beamertemplatearticlebibitems


  \bibitem{lockhart2013}
      Lockhart T, Stergiou N 
      \newblock \doublequoted{New Perspectives in Human Movement Variability}
	\newblock Ann Biomed Eng. 2013.

  \bibitem{cao1997}
      Cao Liangyue
      \newblock \doublequoted{Practical method for determining the minimum embedding dimension of a scalar time series.}
      \newblock Physica D, 110, 43-50, 1997.  

  \bibitem{xochicale2018}
      Xochicale M P
      \newblock \doublequoted{Emotion and Movement Variability: a pilot study}
      \newblock GitHub repo (2018), https:// github.com/mxochicale/emmov-pilotstudy [\href{https:// github.com/mxochicale/emmov-pilotstudy}{\faGithub}]


    \end{thebibliography}
\end{frame}




%%%%%%%%%%%%%%%%%%%%%%%%%%%%%%%%%%%%%%%%%%%%%%%%%%%%%%%%
\closingtitle


%%%%%%%%%%%%%%%%%%%%%%%%%%%%%%%%%%%%%%%%%%%%%%%%%%%%%%%%
\licenseframe{https://github.com/mxochicale/eccc2018}



\end{document}
